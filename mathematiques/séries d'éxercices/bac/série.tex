\documentclass[a4paper]{article}
\usepackage[french]{babel}
\usepackage[utf8]{inputenc}
\usepackage{fancyhdr, mhchem, siunitx, amsmath, amssymb, amsfonts, tikz, amsthm, tkz-tab}
\usepackage{tikz-3dplot, pgfplots}
\usepackage{enumitem, caption}
%\usepackage[margin=4cm]{geometry}
\pagestyle{fancy}
\fancyhf{}
\rhead{\today}
\chead{Mathématiques}
\lhead{Ahmed Chaïbi}
\rfoot{Page numéro \thepage} 
\newcommand{\parasep}{
\begin{center}
   \includegraphics[scale=.5]{../../../FiguresCommunes/parasep.png}
\end{center}}
\begin{document}
\subsection*{Exercice 1.\footnote{C.BJ.} (**)}
On considère l'équation \[
	(E):x^2-7y^2=1
.\] ou les inconnues $x$ et $y$ sont des entiers naturels non nuls.
\begin{enumerate}
	\item \begin{enumerate}[label=(\alph*)]
		\item Comparer $a$ et $b$.\\
		\item Montrer que $1$ est le seul diviseur positif commun à $a$ et à $b$.\\
		\item Démontrer que $a\equiv 1\pmod{7}$ ou $a\equiv -1 \pmod{7}$.
	\end{enumerate}
\item Trouver la solution $(a,b)$ de $(E)$ telle que $b$ soit le plus petit possible.
\item 	\begin{enumerate}[label=(\alph*)]
	\item Démontrer par récurrence qu'il existe un couple $(a_n,b_n)$ d'entiers naturels non nuls solution de $(E)$ et tel que \[
			(8+3\sqrt{7})^n=a_n+b_n\sqrt{7}	.\]
		\item Combien l'équation $(E)$ a-t-elle de solutions?
	\end{enumerate}	
\item 	\begin{enumerate}[label=(\alph*)]
	\item Prouver que pour tout entier naturel non nul $n$ on a \[
			(P_n):(8-3\sqrt{7})^n=a_n-b_n\sqrt{7}.\] 		
	\item En déduire les expressions de $a_n$ et $b_n$ en fonction de $n$.
	\end{enumerate}
\end{enumerate}



\subsection*{Exercice 2.\footnote{C.BJ. modifié.} (***)}
\subsubsection*{Partie I.}
On considère la fonction $f$ définie sur $I=]-1,+\infty[$ par \[
	\begin{cases}
		f(x)&=\frac{\ln(x+1)}{x}, \; \; \; x\neq 0\\
		f(0)&=1
	\end{cases}
.\] 
Soit $\mathcal{C}_f$ sa courbe représentative dans le plan muni d'un repère orthonormé $(O,\vec{i},\vec{j})$.
\begin{enumerate}
	\item Montrer que $f$ est continue sur $I$.
	\item \begin{enumerate}[label=(\alph*)]
		\item Démontrer que pour tout réel $x\in\mathbb{R}_+^*$, puis pour tout réel $x\in]-1,0[$ on a, \[
				0\leq\frac{1}{x}\int_0^x\frac{t^2}{t+1}dt\leq x\int_0^x\frac{1}{1+t}dt
	.\]
\item Démontrer que pour tout réel $x\in I\setminus\{0\}$ on a, \[
		f(x)=1-\frac{1}{2}x-\frac{1}{x}\int_0^x\frac{t^2}{t+1}dt
.\] 
\item Démontrer, alors, que $f$ est dérivable en $0$, déterminer une équation de la tangente à $\mathcal{C}_f$ au point d'abscisse $0$ et étudier la position de celle-ci par rapport à cette tangente.
	\end{enumerate}
\item Soit $g$ la fonction définie sur $I$ par \[
		g(x)=\ln(x+1)-\frac{x}{x+1}
.\] 
Etudier le signe de $g$ pour $x\in I$ et en déduire le sens de variation de $f$.
\item Construire $\mathcal{C}_f$.
\end{enumerate}
\subsubsection*{Partie II.}
\begin{enumerate}
	\item Justifier que pour tout réels $a$ et $b$ de $I$ tels que $a<b$ on a, \[
			(b-a)f(b)\leq\int_a^bf(t)dt\leq(b-a)f(a)
	.\] 
	En utilisant la méthode des rectangles pour $n=5$, en déduire un encadrement de l'aire de la partie du plan délimitée par l'axe des abscisses, la courbe $\mathbb{C}_f$ et les droites $(O,\vec{i}$ et $\Delta:x=1$.\\
	\item Soit $h$ la fonction définie sur $I$ par \[
			h(x)=x+1-(x+1)\ln(x+1)
	.\] 
	\begin{enumerate}[label=(\alph*)]
		\item Dresser le tableau de variation de $h$.
		\item Montrer que pour tout $x\in]-1,-\frac{1}{2}]$ on a, \[
				0\leq f(x)\leq-2\ln(x+1)
		.\] 
	\item En déduire que la fonction \[
			F:x\mapsto\int_x^{-\frac{1}{2}}f(t)dt\] 
			est majorée dans $]-1,-\frac{1}{2}]$.
	\end{enumerate}
\item	On considère la suite $(V_n)_{n\in\mathbb{N}*}$ de terme général \[
		V_n=\int_{-1+\frac{1}{n}}^0f(t)dt
.\] 
Etudier le sens de variation de la suite $(V_n)_{n\in\mathbb{N}*}$ et conclure.
\end{enumerate}
\newpage
\subsection*{Exercice 3.\footnote{Sujet baccalauréat 2008, modifié.} (*)}
Le plan est orienté dans le sens trigonométrique. Soit $OAB$ un triangle isocèle rectangle direct en $O$. On désigne par $I$ le milieu de $[AB]$ et par $C$ et $D$ les symétriques respectifs du point $I$ par rapport à $O$ et à $B$. Soit $f$ la similitude directe qui envoie $A$ sur $D$ et $O$ sur $C$.

 \begin{enumerate}
\begin{minipage}{0.5\textwidth}
	 \item Montrer que $f$ est de rapport $2$ et d'angle $\frac{\pi}{2}$.
	 \item \begin{enumerate}[label=(\alph*)]
		 \item Montrer que $O$ est l'orthocentre du triangle $ACD$.
		 \item Soit $J$ le projeté orthogonal du point $O$ sur $(AC)$. Montrer que $J$ est le centre de la similitude $f$.
		\end{enumerate}
	\item Soit $g$ la similitude indirecte de centre $I$ qui envoie $A$ sur $D$.
		\begin{enumerate}[label=(\alph*)]
			\item Vérifier que $g$ est de rapport $2$ et d'axe $(IC)$. En déduire $g(O)$.
			\item Déterminer la nature (et les caractéristiques) de $g\circ f^{-1}$. 
		\end{enumerate}
\end{minipage}%
\begin{minipage}{0.5\textwidth}
	\centering
	\begin{tikzpicture}[scale=0.6]
	\draw (0,0) node[left]{$O$} -- (3,0) node[right]{$A$} -- (0,3) node[above right]{$B$} -- (0,0);
	\draw (-1.5,-1.5) node[below left]{$C$} -- (1.5,1.5) node[above right]{$I$};
	\draw (-1.5,4.5) node[above left]{$D$} -- (0,3);
	\draw[red] (-1.5,4.5) -- (-1.5,-1.5);
	\draw (-1.5,-1.5) -- (3,0);
	\draw (0.3,-0.9) node[below right]{$J$} -- (0,0);
	\draw (0.48,-0.84) -- (0.42,-0.66) -- (0.247, -0.723);
	\draw (0,0.2) -- (0.2,0.2) -- (0.2,0);
	\draw (1.6414, 1.355857) -- (1.5, 1.217157) -- (1.355857,1.355857);
	\draw (1.5,-0.1) -- (1.5,0.1);
	\draw (-0.1,1.5) -- (0.1,1.5);
	\draw[red] (0.3,-0.9) -- (1.5,1.5);
\end{tikzpicture}
\captionof{figure}{Construction I.}
\end{minipage}
	\item Soient $I'=f(I)$ et $J'=g(J)$. Montrer que les droites $(IJ), (I'J')$ et $(CD)$ sont concourrantes.
\end{enumerate}


\subsection*{Exercice 4.\footnote{Sujet baccalauréat 2009, modifié.} (*)}
Le plan est orienté dans le sens trigonométrique. Soit $ABC$ un triangle isocèle rectangle direct en $A$. On désigne par $I, J, K, H$ et $L$ les milieux respectifs des segments $[AB], [BC], [AC], [AJ]$ et $[JC]$.
\begin{enumerate}
	\item Soit $f$ la similitude directe de centre $J$ qui envoie $A$ sur $K$. Déterminer ses caractéristiques puis déterminer les images de $L$ et $I$.

\begin{minipage}{0.5\textwidth}
	\item On muni le plan du repère $(A,\overrightarrow{AB},\overrightarrow{AC})$. Soit $\phi$ l'application du plan dans lui-même qui à tout point $M$ associe le point $M'$ tels que \[
			z_{M'}=-\frac{1+i}{2} \overline{z_M}+\frac{1+i}{2}
	.\] 
	\begin{enumerate}
		\item Montrer que $\phi$ est une similitude indirecte de centre $C$.
		\item Montrer que $\phi=f\circ S_{(IK)}$.
	\end{enumerate}
\end{minipage}%
\begin{minipage}{0.5\textwidth}
	\centering
\begin{tikzpicture}[scale=1]
	\draw (0,0) node[below left]{$A$} -- (1.5,0) node[below left]{$I$} -- (3,0) node[below right]{$B$} -- (1.5,1.5) node[above right]{$J$} -- (0.75,2.25) node[above right]{$L$} -- (0,3) node[above left]{$C$} -- (0,1.5) node[below left]{$K$} -- (0,0);
	\draw[red] (0.75,0.75) node[above right]{$H$};
	\draw[red, thick] (-0.5,2) -- (2,-0.5);
	\fill[red] (0.75,0.75) circle (0.5pt);
	\fill (0.75,2.25) circle (0.5pt);
	\fill (1.5,1.5) circle (0.5pt);
\end{tikzpicture}
\captionof{figure}{Construction II.}
\end{minipage}
\end{enumerate}
\newpage
\subsection*{Exercice 5.\footnote{I.M.}}
Soit $OAB$ un triangle isocèle rectangle tel que $AB=4$. On note $I$ le milieu de $[AB]$ et $F$ le point défini par $\overrightarrow{OF}=\frac{1}{4}\overrightarrow{OI}$. Soit $\mathcal{P}$ la parabole de foyer $F$ et de sommet $O$. On munit le plan du repère orthonormé $(O,\vec{i},\vec{j})$ tel que $\vec{i}=\frac{1}{2}\overrightarrow{OI}$.

\begin{enumerate}
	\item Montrer que dans ce repère, $\mathcal{P}$ a pour équation $y^2=2x.$ Puis, montrer que $\mathcal{P}$ passe par $A$ et $B$.



	\begin{minipage}{0.45\textwidth}
	\item La tangente à $\mathcal{P}$ en $A$ coupe $(OI)$ en un point $J$. Montrer que $O$ est le milieur $[IJ]$.
	\item Soit $M(x_1,y_1)$ un point de $\mathcal{P}$ distinct de $O$. La perpendiculaire à $(OM)$ en $O$ recoupe $\mathcal{P}$ en un point $N(x_2,y_2)$. Soient les réel $(a,b)$ pour lesquelles $(MN)$ a pour équation $y=ax+b$. 

		Montrer que $y_1$ et $y_2$ sont les solutions de l'équation $y^2-2ay-2b=0$ puis que $b=2$. 

	\end{minipage}%
\begin{minipage}{0.55\textwidth}
	\centering
	\begin{tikzpicture}
	\coordinate (O) at (0,0);
	\coordinate (A) at (2,-2);
	\coordinate (B) at (2,2);
	\coordinate (I) at (2,0);
	\coordinate (F) at (1,0);
	\draw (O) node[below left]{$O$} -- (A) node[below]{$A$} -- (B) node[above]{$B$} -- (O) -- (I) node[right]{$I$};
	\draw (-2.75,0) -- (O);
	\draw (1.8,0) -- (1.8,0.2) -- (2,0.2);
	\draw (0.1415,0.1415) -- (0.2828,0) -- (0.1415,-0.1415);
	\draw[domain=0:3, red, thick, samples=50] plot (\x,{sqrt(2*\x)});
	\draw[domain=0:3, red, thick,  samples=50] plot (\x,{-sqrt(2*\x)});
	\draw[domain=-2.75:3, samples=50] plot (\x,{-0.5*\x-1});
	\draw (-2.15,0) node[below]{$J$};
	\draw[->] (O) -- (1,0) node[midway, above]{$\vec{i}$};
	\draw[->] (O) -- (0,1) node[midway, left]{$\vec{j}$};
	\draw (0.5,0) node[below]{$F$};
	\fill (0.5,0) circle (1pt);
	\draw (2.75,2.345) node[above, red]{$\mathcal{P}$};
	\draw (2,1) node[left]{$2$};
	\end{tikzpicture}
	\captionof{figure}{Construction III.}
\end{minipage}
	\item En déduire que lorsque $M$ varie sur $\mathcal{P}$, la droite  $(MN)$ passe par un point fixe.

\end{enumerate}
\subsection*{Exercice 6.\footnote{I.M.} (**)}
Soit $f$ une fonction décroissante sur $[0,1]$. On considère la suite $(u_n)_{n\in\mathbb{N}^*}$ de terme général $$u_n=\frac{1}{n}\sum_{i=1}^nf\Big(\frac{i}{n}\Big).$$
\begin{enumerate}
	\item Montrer que pour tout $n\in\mathbb{N}^*$ et pour tout $i\in\{0,1,\ldots,n-1\}$ on a:  \[
			\frac{1}{n}f\Big(\frac{i+1}{n}\Big)\leq\int_{\frac{i}{n}}^{\frac{i+1}{n}}f(t)dt\leq\frac{1}{n}f\Big(\frac{i}{n}\Big)
		.\] 
	\item En déduire la limite de $(u_n)_{n\in\mathbb{N}^*}$.
\item Calculer,
	\[
		\lim_{n\to+\infty}n\sum_{i=1}^n\frac{1}{(i+n)^2}\; \text{ et } \; \lim_{n\to+\infty}\frac{1}{\sqrt{n}^3}\sum_{i=1}^{n}\sqrt{n-i}
	.\] 
\end{enumerate}
\end{document}
