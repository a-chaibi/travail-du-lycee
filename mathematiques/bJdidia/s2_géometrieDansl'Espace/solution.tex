\documentclass{article}
\usepackage[french]{babel}
\usepackage[utf8]{inputenc}
\usepackage[T1]{fontenc}
\usepackage{fancyhdr, amsmath, amssymb, amsfonts, tikz, amsthm, tkz-tab}
\usepackage{tikz-3dplot}
\usepackage{enumitem, caption}
\usetikzlibrary{patterns}
%\usepackage[margin=4cm]{geometry}
\pagestyle{fancy}
\fancyhf{}
\rhead{\today}
\chead{Solution proposée}
\lhead{Ahmed Chaïbi}
\rfoot{Page numéro \thepage}
\DeclareMathOperator{\distan}{d}
\begin{document}
\section*{Q.C.M}
\subsection*{Exercice 1.}
\begin{minipage}[c]{0.3\textwidth}	
\begin{enumerate}[label=(\alph*)]
	\item Faux, $$\overrightarrow{IC}\cdot\overrightarrow{IO}=\frac{3}{4}a^2.$$
	\item Vrai.
	\item Faux, $OB=OC.$
	\end{enumerate}
\end{minipage}%
\begin{minipage}[c]{0.7\textwidth}	
	\centering
	% Construction I
	\tdplotsetmaincoords{100}{50}
	\begin{tikzpicture}[tdplot_main_coords, scale=0.7]
		% Le cube
		\draw[dashed] (0,0,0) node[anchor=north east]{$A$} -- (3,0,0) node[anchor=north]{$D$} -- (3,3,0) node[anchor=north west]{$C$};
		\draw (3,3,0) -- (0,3,0) node[anchor= north]{$B$}; 
		\draw (0,3,0) -- (0,0,0) node[midway,anchor=north]{$a$} -- (0,0,3) node[anchor= south east]{$E$} -- (3,0,3) node[anchor=south]{$H$} -- (3,3,3) node[anchor=south west]{$G$} -- (0,3,3) node[anchor=south]{$F$} -- (0,0,3);
		\draw[dashed] (3,0,0) -- (3,0,3);
		\draw (3,3,0) -- (3,3,3);
		\draw (0,3,0) -- (0,3,3);

		% Le point O
		\draw[dotted] (0,0,0) -- (3,0,3);
		\draw[dotted] (0,0,3) -- (3,0,0);
		\draw (1.5,0,1.5) node[right, red]{$O$};
		\fill[red] (1.5,0,1.5) circle (1pt);
		% Le point I
		\draw (0,1.5,3) node[below, red]{$I$};
		\fill[red] (0,1.5,3) circle (1.2pt);
	
	\end{tikzpicture}
	\captionof{figure}{Construction I.}
	% Fin de construction I
\end{minipage}%
\subsection*{Exercice 2.}
\begin{minipage}[c]{0.3\textwidth}
\begin{enumerate}[label=(\alph*)]
	\item Vrai.
	\item Faux, $$\overrightarrow{SC}\cdot\overrightarrow{SD}=23.$$
	\item Faux, $$\mathcal{V}_{(SABCD)}=24 \; \; \; (u.v).$$
\end{enumerate}
\end{minipage}%
\begin{minipage}[c]{0.7\textwidth}	
	\centering
	% Construction II
	\tdplotsetmaincoords{100}{75}
	\begin{tikzpicture}[tdplot_main_coords, scale=0.7]
		% La pyramide
		\draw[dashed] (0,0,0) node[anchor=east]{$A$} -- (3,0,0)  node[midway, anchor=south]{$3$};
		\draw[dashed] (3,0,0) node[anchor=south west]{$B$}-- (3,4,0) node[midway, anchor=south west]{$4$};
		\draw (3,4,0) node[anchor=north west]{$C$} -- (-3,4,0) node[midway, anchor=north west]{$6$};
		\draw (-3,4,0) node[anchor=north]{$D$} -- (0,0,0);
		\draw (0,0,0) -- (0,0,4) node[anchor=east, midway]{$4$};
		\draw (0,0,4) node[anchor=south]{$S$} -- (-3,4,0);
		\draw[dashed] (3,0,0) -- (0,0,4);
		\draw (0,0,4) -- (3,4,0);
		\draw[red] (0.4,0,0) -- (0.4,0,0.4) -- (0,0,0.4);
		\draw[red] (2.6,0,0) -- (2.6,0.4,0) -- (3,0.4,0);
		\draw[red] (2.6,4,0) -- (2.6,3.6,0) -- (3,3.6,0);
	\end{tikzpicture}
	\captionof{figure}{Construction II.}
	% Fin de construction II
\end{minipage}\\
\\
\\
\begin{minipage}[t]{0.6\textwidth}
\subsection*{Exercice 3.}
\begin{enumerate}[label=(\alph*)]
	\item Vrai.
	\item Vrai.
	\item Faux, \begin{align*}
			\mathcal{A}_{(ABCD)}&=||\overrightarrow{AB}\wedge\overrightarrow{AD}||\\ 
					&=3\sqrt{10} \; \; (u.a).
	\end{align*}
\end{enumerate}
\end{minipage}%
\begin{minipage}[t]{0.4\textwidth}
\subsection*{Exercice 5.}
\begin{enumerate}[label=(\alph*)]
	\item Faux, le vecteur nul est colinéaire à tout vecteur.
	\item Vrai.
	\item Vrai.
\end{enumerate}
\end{minipage}
\subsection*{Exercice 4.}
\begin{minipage}[c]{0.5\textwidth}
\begin{enumerate}[label=(\alph*)]
	\item Vrai.
	\item Vrai.
	\item Faux, \begin{align*}
			\mathcal{A}_{(BDM)}&=\frac{1}{2}||\overrightarrow{MB}\wedge\overrightarrow{MD}||\\ 
					   &=\frac{\sqrt{6}}{4} \; \; (u.a).
	\end{align*}
\end{enumerate}
\end{minipage}%
\begin{minipage}[c]{0.7\textwidth}	
	\centering
	% Construction III
	\tdplotsetmaincoords{95}{72}
	\begin{tikzpicture}[tdplot_main_coords]
		\draw[dashed] (0,0,0) node[anchor=north east]{$A$} -- (3,0,0) node[anchor=east]{$B$} -- (3,3,0) node[anchor=north west]{$C$};
		\draw (3,3,0) -- (0,3,0) node[anchor= north]{$D$}; 
		\draw (0,3,0) -- (0,0,0) node[midway,anchor=north]{$1$} -- (0,0,3) node[anchor= south east]{$E$} -- (3,0,3) node[anchor=south]{$F$} -- (3,3,3) node[anchor=south west]{$G$} -- (0,3,3) node[anchor=south]{$H$} -- (0,0,3);
		\draw[dashed] (3,0,0) -- (3,0,3);
		\draw (3,3,0) -- (3,3,3);
		\draw (0,3,0) -- (0,3,3);
		\draw (0,0,1.5) node[anchor=east]{$M$};
		\draw [pattern=north west lines, pattern color=red] (0,0,1.5) -- (0,3,0) -- (3,0,0);
		\draw[red, thick] (0,0,1.5) -- (0,3,0) -- (3,0,0) -- (0,0,1.5);
		\draw (0.5,0.5,1) node[anchor=south, blue]{$K$};
		\fill[blue] (0.5,0.5,1) circle (1.5pt);
	\end{tikzpicture}
	\captionof{figure}{Construction III.}
	% Fin de construction II
\end{minipage}

\vspace{1cm}

\begin{minipage}{0.4\textwidth}
\subsection*{Exercice 6.}
\begin{enumerate}[label=(\alph*)]
	\item Vrai.
	\item Vrai.
	\item Faux, car $BA=2\sqrt{5}$.
\end{enumerate}
\end{minipage}%
\begin{minipage}{0.3\textwidth}
	\subsection*{Exercice 7.}
	\begin{enumerate}[label=(\alph*)]
	\item Vrai.
	\item Faux.
	\item Faux.
\end{enumerate}
\end{minipage}%
\begin{minipage}{0.3\textwidth}
	\subsection*{Exercice 8.}
	\begin{enumerate}[label=(\alph*)]
	\item Faux.
	\item Faux.
	\item Vrai.
\end{enumerate}
\end{minipage}


\vspace{1cm}


\begin{minipage}{0.5\textwidth}
	\subsection*{Exercice 9.}
	\begin{enumerate}[label=(\alph*)]
	\item Vrai.
	\item Faux.
	\item Faux.
\end{enumerate}
\end{minipage}%
\begin{minipage}{0.5\textwidth}
	\subsection*{Exercice 10.}
	\begin{enumerate}[label=(\alph*)]
	\item Vrai.
	\item Vrai.
	\item Vrai.
\end{enumerate}
\end{minipage}
\section*{Problèmes.}
\subsection*{Exercice 1.}
\begin{enumerate}
	\item	Un vecteur est unitaire si sa norme vaut $1.$	\begin{align*}
			||\vec{u}||&=\sqrt{(\frac{2}{3})^2+(-\frac{1}{3})^2+(-\frac{2}{3})^2}=1\\
			||\vec{v}||&=\sqrt{(\frac{2}{3})^2+(\frac{2}{3})^2+(\frac{1}{3})^2}=1
		\end{align*}
		donc, $\vec{u}$ et $\vec{v}$ sont unitaires.
		$$\vec{u}\cdot\vec{v}=\begin{bmatrix}\frac{2}{3}\\ -\frac{1}{3}\\ -\frac{2}{3} \end{bmatrix}\cdot\begin{bmatrix}\frac{2}{3}\\ \frac{2}{3} \\ \frac{1}{3}\end{bmatrix}=\frac{2}{3}\times\frac{2}{3}-\frac{1}{3}\times\frac{2}{3}-\frac{2}{3}\times\frac{1}{3}\\=0.$$
		Ainsi, $\vec{u}$ et $\vec{v}$ sont orthogonales.
	\item  \[
			\vec{w}=\vec{u}\wedge\vec{v}=\begin{bmatrix}\frac{2}{3}\\ -\frac{1}{3}\\ -\frac{2}{3} \end{bmatrix}\wedge\begin{bmatrix}\frac{2}{3}\\ \frac{2}{3} \\ \frac{1}{3}\end{bmatrix}=\begin{bmatrix}\frac{1}{3} \\ -\frac{2}{3} \\ \frac{2}{3}\end{bmatrix}
	.\] 
\end{enumerate}
\subsection*{Exercice 2.}
\begin{enumerate}
	\begin{minipage}{0.6\textwidth}
	\item \begin{enumerate}
		\item \[
				\begin{cases}
					\frac{x_I+x_L}{2}=\frac{1+1}{2}=1=x_A\\
					\frac{y_I+y_L}{2}=\frac{0+0}{2}=0=y_A\\
					\frac{z_I+z_L}{2}=\frac{0+1}{2}=\frac{1}{2}=z_A
				\end{cases}
		.\] 	
		Ainsi $A$ est le milieu du segement $[IL]$.
		$$
			\overrightarrow{KB}=\begin{bmatrix}
				0\\
				\frac{2}{3}\\
				0
				\end{bmatrix}=\frac{2}{3}\begin{bmatrix}
				0\\
				1\\
				0
			\end{bmatrix}=\frac{2}{3}\overrightarrow{KN}
	.$$
\item \[
		\vec{u}=\overrightarrow{OA}\wedge\overrightarrow{OB}=\begin{bmatrix}1\\0\\\frac{1}{2}\end{bmatrix}\wedge\begin{bmatrix}0\\\frac{2}{3}\\1\end{bmatrix}=\begin{bmatrix}-\frac{1}{3}\\-1\\\frac{2}{3}\end{bmatrix}
.\] 
	\end{enumerate}
\end{minipage}%
\begin{minipage}{0.4\textwidth}
	\centering
	\tdplotsetmaincoords{110}{65}
	\begin{tikzpicture}[tdplot_main_coords, scale=0.8]
		\coordinate (O) at (0,0,0);
		\coordinate (I) at (0,3,0);
		\coordinate (R) at (3,3,0);
		\coordinate (J) at (3,0,0);
		\coordinate (A) at (0,3,1.5);
		\coordinate (B) at (1,0,3);
		\coordinate (K) at (0,0,3);
		\coordinate (L) at (0,3,3);
		\coordinate (M) at (3,3,3);
		\coordinate (N) at (3,0,3);

		\draw (O) node[anchor=north east, red]{$O$} -- (I) node[anchor=north]{$I$} -- (R) node[anchor=north west]{$R$} -- (M) node[anchor=south west]{$M$} -- (L) node[anchor=south]{$L$} -- (I);
		\draw (K) node[anchor=south east, red]{$K$} -- (L);
		\draw (M) -- (N) node[anchor=south]{$N$} -- (B) node[anchor=south east, red]{$B$};
	\draw[dashed] (O) -- (J) node[anchor=south west]{$J$} -- (R);
	\draw[dashed] (N) -- (J);
	\draw[red, thick] (B) -- (K) -- (O) -- (A) node[anchor=west]{$A$} -- (K);
	\draw[red, thick, dashed] (O) -- (B) -- (A);
	\draw (1,3,3) node[anchor=north west]{$C$};
	\fill (1,3,3) circle (1pt);

	\end{tikzpicture}
	\captionof{figure}{Construction IV.}
\end{minipage}

\item \begin{enumerate}
	\item $$\mathcal{A}_{(OAB)}=\frac{1}{2}||\overrightarrow{OA}\wedge\overrightarrow{OB}||=\frac{1}{2}\sqrt{\Big(-\frac{1}{3}\Big)^2+(-1)^2+\Big(\frac{2}{3}\Big)^2}=\frac{\sqrt{14}}{6} \; \; \; \; (u.a).$$
	\item $$\overrightarrow{OC}\cdot(\overrightarrow{OA}\wedge\overrightarrow{OB})=\overrightarrow{OC}\cdot\vec{u}=\begin{bmatrix}1\\\frac{1}{3}\\1\end{bmatrix}\cdot\begin{bmatrix}-\frac{1}{3}\\-1\\\frac{2}{3}\end{bmatrix}=-\frac{1}{3}\times1-1\times \frac{1}{3}+\frac{2}{3}\times1=0.$$
		Ainsi $\overrightarrow{OA}, \overrightarrow{OB}$ et $\overrightarrow{OC}$ sont coplanaires et par suite $C\in\mathcal{P}$.
\end{enumerate}
\item \begin{enumerate}
	\item \[
		\mathcal{V}_{(OABC)}=\frac{1}{6}\Bigg|\overrightarrow{OK}\cdot(\overrightarrow{OA}\wedge\overrightarrow{OB})\Bigg|
		=\frac{1}{6}\Bigg|\begin{bmatrix}0\\0\\1\end{bmatrix}\cdot\begin{bmatrix}-\frac{1}{3}\\-1\\\frac{2}{3}\end{bmatrix}\Bigg|
		=\frac{1}{6}\Bigg|-\frac{1}{3}\times0-1\times0+\frac{2}{3}\times1\Bigg|=\frac{1}{9} \; \; \; (u.v)
.\] 
\item Or,  \[
	\mathcal{V}_{(OABC)}=\frac{1}{3}\mathcal{A}_{(OAB)}\distan(K,\mathcal{P}),\] 
donc, 
\[
	\distan(K,\mathcal{P})=\frac{3\mathcal{V}_{(OABC)}}{\mathcal{A}_{(OAB)}}= \frac{3\times \frac{1}{9}}{\frac{\sqrt{14}}{16}}=\frac{\sqrt{14}}{7}.
.\] 
\end{enumerate}
\end{enumerate}
\subsection*{Exercice 3.}
\begin{enumerate}
	\item Le vecteur $\vec{n}\begin{bmatrix}1\\-1\\1\end{bmatrix}$ est normal à $\mathcal{P}$ donc $\mathcal{P}:x-y+z+d=0$. Or, le point $A(1,0,1)\in\mathcal{P}$ donc $d=-2$ et,
		\begin{equation}
			\mathcal{P}:x-y+z-2=0\label{eq:planP}
.\end{equation}
Et puis,
\begin{align}
	S:&x^2+y^2+z^2-2x-8z+13=0\label{eq:sphèreS}\\
	&(x-1)^2+y^2+(z-4)^2=2^2
	\nonumber
\end{align}
donc $S$ est la sphère de centre  $C(1,0,4)$ et de rayon $R=2$.
 \[
	 \distan(C,\mathcal{P})=\frac{\Big|1-0+1-2\Big|}{\sqrt{1^2+(-1)^2+1^2}}=\sqrt{3}
.\]
Dès lors, $\distan(C,\mathcal{P})<R$ et $\mathcal{P}$ et $S$ sont sécants.
\item Soit  $\mathcal{D}$ la perpendiculaire à $\mathcal{P}$ menée par $C(1,0,4)$ et $H\in\mathcal{D}\cap\mathcal{P }$. Alors $ \vec{n}$ est directeur de $\mathcal{D}$ et pour $\alpha$ un paramètre réel
	\begin{equation}
		\mathcal{D}:\begin{cases}
			x=\alpha+1\\
			y=-\alpha\\
			z=\alpha+4
		\end{cases}.\label{eq:droiteD}
	\end{equation}
	Comme $H$ est l'intersection de $\mathcal{P}$ et $\mathcal{D}$, ses coordonnées satisfont aux équations \eqref{eq:planP} et \eqref{eq:droiteD} donc $\alpha_H+1-(-\alpha_H)+\alpha_H+4-2=0$ puis $\alpha_H=2$ et $H(2,-1,5)$. $S\cap\mathcal{P}$ est un cercle de centre $H$ et de rayon $r=\sqrt{R^2-CH^2}$. On a $\overrightarrow{CH}=\begin{bmatrix}1\\-1\\1\end{bmatrix}$ donc $CH=\sqrt{3}$ et de là $r=1$.
\item Soit $Q$ un plan tangent à $S$ et parallèle à $\mathcal{P}$ donc $\vec{n}$ est normal à $Q$ et $\distan(C,Q)=R$ alors $Q:x-y+z+\beta=$ et
	$$\frac{\Big|1-0+4+\beta\Big|}{\sqrt{1^2+(-1)^2+1^2}}=2$$
	par suite $\Big|\beta+5\Big|=2\sqrt{3}$ et $\beta\in\{2\sqrt{3}-5, -2\sqrt{3}-5\}$. Ainsi $Q\in\{P_1,P_2\}$ avec
	\begin{align*}
		P_1&:x-y+z+2\sqrt3-5=0\\
		P_2&:x-y+z-2\sqrt3-5=0
	\end{align*}
\item On a, \begin{equation}
		\Delta:\begin{cases}
			x=t+1\\
			y=-t\\
			z=t+3
		\end{cases}
		.\label{eq:droiteDelta}\end{equation}
Ainsi $\vec{n}$ est directeur de $\Delta$, et comme il est normal à $\mathcal{P}$ il en découle que $\Delta\perp\mathcal{P}$.

Soit  $M(x,y,z)$ un point de $\Delta\cap S$, donc ses coordonnées satisfont au systèmes \eqref{eq:sphèreS} et \eqref{eq:droiteDelta} donc,  \[
	(t+1)^2+(-t)^2+(t+3)^2-2(t+1)-8(t+3)+13=0
.\] 
Ou encore, $3t^2-2t-3=0$ dont les solutions sont $t_1=\frac{1+\sqrt{10}}{3}$ et $t_2=\frac{1-\sqrt{10}}{3}$ qui correspondent au points,

$$\begin{cases}
	M_1\Big(\frac{1+\sqrt{10}}{3}+1,-\frac{1+\sqrt{10}}{3},\frac{1+\sqrt{10}}{3}+3\Big)\\
	M_2\Big(\frac{1-\sqrt{10}}{3}+1,-\frac{1-\sqrt{10}}{3},\frac{1-\sqrt{10}}{3}+3\Big)
\end{cases}.
$$
Bien entendu, $S\cap\mathcal{D}=\{M_1,M_2\}$.
\end{enumerate}
\subsection*{Exercice 4.}
\begin{enumerate}
	\item	L'équation de $S$ s'écrit,
		\begin{equation}
			S:(x-2)^2+(y+1)^2+z^2=3^2.\label{eq:sphèreII}
	\end{equation} 
	Alors  $S$ est une sphère de centre $W(2,-1,0)$ et de rayon $R=3$.
\item  \[
		\overrightarrow{WA}\wedge\vec{u}=\begin{bmatrix}0\\0\\1\end{bmatrix}\wedge\begin{bmatrix}1\\-2\\-1\end{bmatrix}=\begin{bmatrix}2\\1\\0\end{bmatrix}
.\] 
\[
	\distan(W,\mathbb{D})=\frac{\Big|\Big|\overrightarrow{WA}\wedge\vec{u}\Big|\Big|}{||\vec{u}||}=\frac{\sqrt{2^2+1^2+0^2}}{\sqrt{1^2+(-2)^2+1^2}}=\frac{\sqrt{30}}{6}
.\] 
On a $\vec{u}\begin{bmatrix}1\\-2\\1\end{bmatrix}$ est directeur de $\mathcal{D}$ et $A(2,-1,1)\in\mathcal{D}$ donc, pour $t$ un paramètre réel.
\begin{equation}
	\mathcal{D}:\begin{cases}
		x=t+2\\
		y=-2t-1\\
		z=-t+1
	\end{cases}\label{eq:droiteII}
	\end{equation}
	Soit $M(x,y,z)$ un point de l'intersection de $S$ et $\mathcal{D}$. Donc ses coordonnées satisfont à l'équation \eqref{eq:sphèreII} aussi bien qu'au système \eqref{eq:droiteII}. Alors,
	\[
		(t+2-2)^2+(-2t-1+1)^2+(-t+1)^2=3^2
	.\] 
	Ou encore, $6t^2+8$ donc $t\in\Big\{\frac{2\sqrt{3}}{3},-\frac{2\sqrt{3}}{3}\Big\}$. On trouve ainsi les points points suivants,
	 \[
		 \begin{cases}
			 M_1\Big(\frac{2\sqrt{3}}{3}+2,-\frac{4\sqrt{3}}{3}-1,-\frac{2\sqrt{3}}{3}+1\Big)\\
			 M_2\Big(-\frac{2\sqrt{3}}{3}+2,-\frac{4\sqrt{3}}{3}-1,\frac{2\sqrt{3}}{3}+1\Big)\\
			 S\cap\mathcal{D}
		 \end{cases}.
	\] 
\item $$\distan(W,P_m)=\frac{\Big|0-4+m|}{\sqrt{0^2+1^2+(-1)^2}}=\frac{\big|m-4\big|}{\sqrt{2}}.$$
	L'équation  $\distan(W,P_m)=R$ équivaut à  $m^2-8m-2=0$. Ainsi,
	\begin{itemize}
		\item Si $m>4+3\sqrt{2}$ ou $m<4-3\sqrt{2}$, alors  $S\cap \mathcal{P}_m=\emptyset$.
		\item Si $m\in\Big\{4+3\sqrt{2}, 4-3\sqrt{2}\Big\}$, alors $S\cap \mathcal{P}_m$ est un point.
	\item Sinon, $S\cap \mathcal{P}_m$ est un cercle de rayon $\sqrt{3^2-\distan(W,\mathcal{P}_m)}$ et dont le centre est le projeté orthogonal de $W$ sur $\mathcal{P}_m$. 
\end{itemize}
	\item On a $0\in]4-3\sqrt{2},4+3\sqrt{2}[$ donc $S\cap\mathcal{P}_0$ est un cercle. Soit  $\Delta$ la perpendiculaire à  $\mathcal{P}_0$ passante par $W(2,1,0)$. Comme $\mathcal{P}_0:y-z=0$, on a $\vec{w}\begin{bmatrix}0\\1\\-1\end{bmatrix}$ est normal à $\mathcal{P}_0$ et par suite directeur à $\Delta$. Ainsi, pour  $\alpha$ un paramètre réel,
		\begin{equation}
			\Delta:\begin{cases}
				x=2\\
				y=\alpha-1\\
				z=-\alpha
			\end{cases}\label{eq:delta}
		\end{equation}
		On répète le même algorithme, $H\in\Delta\cap\mathcal{P}_0$ donc  $y_H-zH=0$ et en remplaçant par les expressions de \eqref{eq:delta}, on trouve  $\alpha=\frac{1}{2}$ et $H(2,-\frac{1}{2},-\frac{1}{2})$.
		Ainsi $S\cap\mathcal{P}_0$ est le cercle de centre $H$ et de rayon $r=\sqrt{3^2-WH^2}$. Or $\overrightarrow{WH}\begin{bmatrix}0\\\frac{1}{2}\\-\frac{1}{2}\end{bmatrix}$ donc $WH=\sqrt{\frac{1}{4}+\frac{1}{4}}=\frac{1}{\sqrt{2}}$. Ainsi, $r=\frac{\sqrt{34}}{2}$.
	\item On $O\in\mathcal{P}_0$ donc $\mathcal{P}_0'=h_{(O,2)}(\mathcal{P}_0)=\mathcal{P}_0$ et  $S$ est une sphère de centre  $W(2,-1,0)$ et de rayon $3$ donc  $S'=h_{(O,2)}(S)$ est une sphère de centre $W'(4,-2,0)$ et de rayon $2\times3=6$. Elle est d'équation, \[
			S':(x-4)^2+(y+2)^2+z^2=36
	.\] 
\end{enumerate}
\end{document}
