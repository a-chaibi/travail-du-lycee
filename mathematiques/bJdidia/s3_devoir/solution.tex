\documentclass[a4paper]{article}
\usepackage[french]{babel}
\usepackage[utf8]{inputenc}
\usepackage{fancyhdr, mhchem, siunitx, amsmath, amssymb, amsfonts, tikz, amsthm, tkz-tab}
\usepackage{tikz-3dplot, pgfplots}
\usepackage{enumitem, caption}
\usepackage{csquotes}
%\usepackage[margin=4cm]{geometry}
\usepackage[symbol]{footmisc} 
\renewcommand{\thefootnote}{\fnsymbol{footnote}} 
\pagestyle{fancy}
\fancyhf{}
\rhead{\today}
\chead{Solution proposée}
\lhead{Ahmed Chaïbi}
\rfoot{Page numéro \thepage}
\newcommand{\angorio}[2]{\widehat{(\overrightarrow{#1}, \overrightarrow{#2})}}
\newtheorem*{theorem}{Théorème}
\DeclareMathOperator{\pgcd}{pgcd}
\begin{document}
\subsection*{Exercice 1.}
\begin{enumerate}
	\item Soit \[
			E:\frac{x^2}{a^2}+\frac{y^2}{b^2}=1
	\] 
	l'équation de $E$ et $c$ tel que $b^2+c^2=a^2$. La directrice $\mathcal{D}$ de $E$ a pour équation  $\mathcal{D}:x=\frac{a^2}{c}$ et son excentricité $e=\frac{c}{a}$. Alors,
	\[
		\begin{cases}
			\frac{a^2}{c}=\frac{25}{4}\\
			\frac{c}{a}=\frac{4}{5}
		\end{cases}
	.\] 
	En multipliant les deux équations, on trouve $a=5$ et on tire de la deuxième $c=4$. On calcule $b=\sqrt{a^2-c^2}=\sqrt{5^2-4^2}=3$. Ainsi,
	\[
	E:\frac{x^2}{25}+\frac{y^2}{9}=1
	.\] 
\item L'expression de la fonction complexe associée à $f$ est de la forme $z'=az+b$ avec $\begin{cases}a=i\\b=1-i\end{cases}$ donc $f$ est une similitude directe de rapport $|a|=1$, d'angle $\arg a\equiv\frac{\pi}{2}\pmod{2\pi}$ et de centre $I$ d'affixe $\frac{b}{1-a}=1$. Ainsi $f=R_{(I,\frac{\pi}{2})}$.
\item	\begin{enumerate}
\item $f$ est une similitude donc elle conserve les formes et par suite l'image de  $E$ par $f$ est une ellipse. De plus, toute similitude conserve les rapports, donc $f$ conserve $e=\frac{c}{a}$ qui est caractéristique de la forme de l'ellipse et comme $f$ est de rapport $1$, $E$ et $E'$ sont isometriques.
\item Le centre de $E$ est $O$ d'affixe $0$ donc celui de $E'$ est $f(O)$ d'affixe $z_O'=1-i$. Les foyers de $E$ sont $F_1$ et $F_2$ d'affixes respectifs $4$ et $-4$ donc ceux de $E'$ sont $f(F_1)$ et $f(F_2)$ d'affixes respectifs $1+3i$ et $1-5i$.	Les sommets de $E$ sont $S_1,S_2,S_3$ et $S_4$ d'affixes respectifs $5,-5,3i$ et $-3i$ donc ceux de $E'$ sont $f(S_1),f(S_2),f(S_3)$ et $f(S_4)$ d'affixes respectifs $1+4i,1-6i,-2-i$ et $4-i$.
	\end{enumerate}
\end{enumerate}
\begin{figure}[h!]
\centering	
\begin{tikzpicture}[scale=0.8]
	\draw (-5.5,0)  -- (5.5,0);
	\draw (0,-3.5) -- (0,3.5);
	\draw[domain=-5:5,red, thick, samples=50] plot (\x,{3*sqrt(1-0.04*\x*\x)});
	\draw[domain=-5:5,red, thick, samples=50] plot (\x,{-3*sqrt(1-0.04*\x*\x)});
	\draw[thick, blue, ->] (0,0) -- (0,1) node[midway, left]{$\vec{j}$};
	\draw[thick, blue, ->] (0,0) -- (1,0) node[midway, above]{$\vec{i}$};
	\draw[<->, blue] (-5,-1) -- (-5,1);
	\draw[<->, blue] (5,-1) -- (5,1);
	\draw[<->, blue] (-1,3) -- (1,3);
	\draw[<->, blue] (-1,-3) -- (1,-3);
	\draw (0,0) node[below left]{$O$};
	\draw (3,0) node[above]{$F_1$};
	\draw (-3,0) node[above]{$F_2$};
	\draw (3,0) circle (1pt);
	\draw (-3,0) circle (1pt);
	\draw (5,0) node[above right]{$S_1$};
	\draw (-5,0) node[above left]{$S_2$};
	\draw (0,3) node[above left]{$S_3$};
	\draw (0,-3) node[above left]{$S_4$};
\end{tikzpicture}
\caption{Construction I.}
\end{figure}
\newpage
\subsection*{Exercice 2.}
\begin{enumerate}
	\item
		\begin{enumerate}
			\item	\begin{theorem}[de Bézout]
			Deux entiers relatifs $a$ et $b$ sont premiers entre eux si, et seulement s'il existe deux entiers relatifs $u$ et $v$ tels que $$au + bv = 1.$$
		\end{theorem}
		Comme $8$ et $5$ sont premiers entre eux, l'équation $(E):8x+5y=1$ admet une solution. $(2,-3)$ est une solution particulière de $(E)$.
	\item  Si $(x,y)$ est solution de $(E)$ alors,
	\begin{align*}
		8x+5y&=1\\
		8x+5y&=8\times2-3\times5\\
		8(x-2)&=-5(y+3)
	\end{align*}
	alors, $5$ divise $8(x-2)$. Or, $8$ et $5$ sont premiers entre eux, donc, d'après la lemme de Gauss $5$ divise, $x-2$ et par suite il existe $k\in\mathbb{Z}$ vérifiant $x=5k+2$. On obtient, alors, $8(5k+2-2)=-5(y+3)$, ou encore $y=-8k-3$.

	Reciproquement, si  $(x,y)=(5k+2,-8k-3)$ avec $k\in\mathbb{Z}$, alors,
	 \[
		 8x+5y=8(5k+2)+5(-8k-3)=40k+16-40k-15=1
	 .\] Et $(x,y)$ est solution de $(E)$. Ainsi,
	 $$S_{\mathbb{Z}^2}=\Big\{(5k+2,-8k-3);k\in\mathbb{Z}\Big\}.$$
\end{enumerate}
\item	\begin{enumerate}
		\item	\begin{align*}
				\text{Si}\;\;\;&\begin{cases}
					x\equiv 1\pmod 8\\
					x\equiv 2\pmod 5
				\end{cases},\\
				\text{alors,}\;\;\;&\begin{cases}
					x=8\alpha+1\\
					x=5\beta+2
				\end{cases}\\
				\text{alors,}\;\;\;&\begin{cases}
					5x=40\alpha+5\\
					8y=40\beta+16
				\end{cases}\\
					\text{alors,}\;\;\;&2\times 8x-3 \times 5x=-3(40\alpha+5)+2(40\beta+16)\\
					\text{alors,}\;\;\;&x=40(2\alpha-3\beta)+17\\
					\text{alors,}\;\;\;&x\equiv 17 \pmod{40}
			\end{align*}
			avec $\alpha,\beta\in\mathbb{Z}$. Reciproquement, si $x\equiv 17 \pmod{40}$, alors $40$ divise $x-17$. Or $8$ et $5$ divisent $40$, donc $8$ et $5$ divisent $x-17$, alors, 
			\begin{align*}
				\begin{cases}
					x\equiv 17 \pmod 8 \\
					x\equiv 17 \pmod 5
				\end{cases}\\
				\begin{cases}
					x\equiv 1 \pmod 8\\
					x\equiv 2 \pmod 5
				\end{cases}.
			\end{align*}
			Ainsi, \[
				S_{\mathbb{Z}}=\{40k-17;k\in\mathbb{Z}\}
			.\] 
		\item Si $x$ est solution de  $(S)$, alors, $x\equiv 17 \pmod{40}$. $0\leq 17\leq 40$, donc $17$ est le reste de la division euclidienne de $x$ par $40$.
	\end{enumerate}
\item \begin{enumerate}
	\item En faisant exactement la même démarche qu'on a faite dans la question 2.(a) pour résoudre $(E)$, seulement avec $(200,-300)$ comme solution particulière, on trouve que  \[
			S_{\mathbb{Z}^2}=\Big\{(5k+200,-8k-300);k\in\mathbb{Z}\Big\}
	.\] 
\item Notons par $x$ le nombre de garçons et $y$ celui des filles. L'énoncée se traduit, 
	\begin{align*}
		&\begin{cases}
			8x+5y=100\\
			x\geq 0\\
			y\geq 0
		\end{cases}\\
		&\begin{cases}
			x=5k+200\\
			y=-8k-300\\
			x\geq 0\\
			y\geq 0
		\end{cases} \; ; \; \; \; \;k\in\mathbb{Z}\\
		&\begin{cases}
			k\leq\frac{-75}{2}\\
			k\geq-40
		\end{cases}
	\end{align*}
	Ainsi, $k\in\{-40,-39,-38\}$ et les répartitions possibles sont:
	\begin{itemize}
		\item Aucun garçon et $20$ filles.
		\item $5$ garçons et $12$ filles.
		\item 10 garçons et $4$ filles.
	\end{itemize}
	\end{enumerate}
\end{enumerate}
\end{document}
