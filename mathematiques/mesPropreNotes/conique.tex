\hypertarget{definition}{%
\subsection{Definition}\label{definition}}

Soit \(D\) une droite, \(F\) un point n'appartenant pas à \(D\), et
\(e>0\) un réel. On appelle conique de directrice \(D\), de foyer \(F\)
et d'excentricité \(e\) l'ensemble des points \(M\) du plan vérifiant
\(\frac{MF}{Mh}=e\), avec \(H\) le projeté orthogonal de \(M\) sur
\(D\). Si \(e>1\), la conique est appelée ellipse, si \(e=1\) parabole,
et si \(e>1\) hyperbole. La première n'ayant qu'un seul sommet et un
seul axe de symétrie se distingue des deux derniers qui sont des avec
deux axes de symétrie \emph{coniques à centre}. Il en découle que
l'ellipse et l'hyperbole ont chaqu'un un second couple foyer-directrice
symétrique au premier. On a toujours \(e=\frac{c}{a}\).

\hypertarget{ellipse}{%
\subsubsection{Ellipse}\label{ellipse}}

\$\(\frac{x^2}{a^2}+{y^2}{b^2}=1\)

Pour l'ellipse, le foyer (principal) a pour coordonnées \((c,0)\) avec
\(a^2=b^2+c^2\), et la directrice pour équation \(x=a^2/c\). Les
paramètres \(a\), \(b\) et \(c\) représentent respectivement la moitié
de la longeur \(AA'\) du grand axe, la moitié de la longeur \(BB'\) du
petit axe et la demo-distance focale (distance entre les deux foyer).

\hypertarget{hyperbole}{%
\subsubsection{Hyperbole}\label{hyperbole}}

\$\(\frac{x^2}{a^2}-{y^2}{b^2}=1\)

Pour l'hyperbole, on pose \(c\) telle que \(c^2=a^2+b^2\).

Suspendisse ornare mattis nulla, in placerat orci pretium et. Fusce
molestie sem turpis, auctor eleifend urna viverra id. Suspendisse quis
ultrices mi. Duis dignissim sollicitudin rhoncus. In a enim posuere
turpis commodo tristique. Phasellus elit est, ultrices id dui id,
iaculis semper mauris. Mauris ac risus eu lorem eleifend faucibus. In
sed sollicitudin enim, in ullamcorper arcu. Pellentesque ut mi dictum,
rhoncus dui sed, lobortis lorem. In augue sem, varius a semper at,
rutrum vitae nulla.
