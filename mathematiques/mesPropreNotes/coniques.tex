\documentclass{article}
\usepackage{soul, amsmath, amssymb, amsfonts, xcolor}
\usepackage[french]{babel}
\title{Coniques}
\author{Ahmed Chaïbi}
\date{\today}
\begin{document}
\maketitle
\subsubsection*{Définition d'une conique}
Soit $D$ une droite, $F$ un point n'appartenant pas à $D$, et $e>0$ un réel. On appelle conique de \sethlcolor{pink}directrice $D$, de foyer $F$ et d'excentricité $e$ l'ensemble des points $M$ du plan vérifiant $MF/MH=e$, avec $H$ le projeté orthogonal de $M$ sur $D$. La conique est appellée
\begin{itemize}
	\item ellipse, si $e<1$
	\item parabole, si $e=1$ et
	\item hyperbole, si $e>1$. 
	\end{itemize}
	La première n'ayant qu'un seul sommet et un seul axe de symétrie se distingue des deux derniers qui sont des avec deux axes de symétrie \textit{coniques à centre}. Il en découle que l'ellipse et l'hyperbole ont chaqu'un un second couple foyer-directrice symétrique au premier. 

\vspace{0.3cm}
\hrule
\vspace{0.3cm}

Pour une conique à centre on pose les paramètres suivants,
\begin{itemize}
	\item $a$, la moitié de la longeur $AA'$ du grand axe.
	\item $b$, la moitié de la longeur $BB'$ du petit axe.
		\item $c$, la demi-distance focale (distance entre les deux foyer).
		\end{itemize}
On a toujours, $$e=\frac{c}{a}.$$


\vspace{0.3cm}
\hrule
\vspace{0.3cm}

\subsubsection*{Ellipse}
$$\frac{x^2}{a^2}+\frac{y^2}{b^2}=1$$
Pour l'ellipse, le foyer (principal) a pour coordonnées $(c,0)$ avec $a^2=b^2+c^2$, et la directrice pour équation $x=a^2/c$. Les paramètres $a$, $b$ et $c$ représentent respectivement la moitié de la longeur $AA'$ du grand axe, la moitié de la longeur $BB'$ du petit axe et la demo-distance focale (distance entre les deux foyer).
$$\frac{x^2}{a^2}-\frac{y^2}{b^2}=1$$
Pour l'hyperbole, on pose $c$ telle que $c^2=a^2+b^2$.
Suspendisse ornare mattis nulla, in placerat orci pretium et. Fusce molestie sem turpis, auctor eleifend urna viverra id. Suspendisse quis ultrices mi. Duis dignissim sollicitudin rhoncus. In a enim posuere turpis commodo tristique. Phasellus elit est, ultrices id dui id, iaculis semper mauris. Mauris ac risus eu lorem eleifend faucibus. In sed sollicitudin enim, in ullamcorper arcu. Pellentesque ut mi dictum, rhoncus dui sed, lobortis lorem. In augue sem, varius a semper at, rutrum vitae nulla.
\end{document}
