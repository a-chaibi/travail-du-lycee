\documentclass{article}
\usepackage[french]{babel}
\usepackage{tabularx, minted, algorithm, algpseudocode}
\title{Théorie informatique}
\author{Ahmed Chaïbi}
\begin{document}
\maketitle
\subsubsection*{Exemple élémentaire}
Effectuer une analyse, un algorithme est une traduction pascal du programme intitulé CRYPT qui réalise le cryptage d'un mot donné en permutant le premiers caractère avec le dernier.
  \vspace{0.3cm}
  \hrule
  \vspace{0.3cm}
On commence par l'analyse:
  \vspace{0.3cm}
  \hrule
  \vspace{0.3cm}
  \begin{algorithmic}[1]
\setcounter{ALG@line}{-1}
\State  Nom=CRYPT
\setcounter{ALG@line}{4}
\State 	Résultat=Ecrire(ch)
\setcounter{ALG@line}{3}
\State	ch[1] $\leftarrow$ aux
\setcounter{ALG@line}{2}
\State	ch[long(ch)] $\leftarrow$ ch[1]
\setcounter{ALG@line}{1}
\State	aux $\leftarrow$ ch[long(ch)]
\setcounter{ALG@line}{0}
\State  ch=Donnée("Ecrire une chaîne de caractères: ")
\setcounter{ALG@line}{5}
\State	Fin CRYPT
  \end{algorithmic}
  \vspace{0.3cm}
  \hrule
  \vspace{0.3cm}



Ensuite, l'algorithme
  \vspace{0.3cm}
  \hrule
  \vspace{0.3cm}
  \begin{minipage}{0.32\textwidth}
  \begin{algorithmic}[1]
\setcounter{ALG@line}{-1}
\Statex  Début CRYPT
\State	Lire(ch)
\State	aux $\leftarrow$ ch[long(ch)]
\State	ch[long(ch)] $\leftarrow$ ch[1]
\State	ch[1] $\leftarrow$ aux
\State  Ecrire(ch)
\State	Fin CRYPT
  \end{algorithmic}
   \end{minipage}%
   \begin{minipage}{0.68\textwidth}
	   \begin{tabular}{|c|m{1.8cm}|m{3.25cm}|}
		   \hline
\multicolumn{3}{|c|}{Tableau de déclaration des objets} \\
\hline
Objet & Type & Rôle \\
\hline\hline
ch & Chaîne de caractères & La chaîne à crypter \\
\hline
aux & Chaîne de caractères & Sert à sauvegarder le premier caractère de ch\\
\hline
\end{tabular}
   \end{minipage}
  \vspace{0.3cm}
  \hrule
  \vspace{0.3cm}





Et, enfin, la traduction Pascal.
  \vspace{0.3cm}
  \hrule
  \vspace{0.3cm}
\begin{minted}{pascal}
program CRYPT;
var
ch,aux:string;
BEGIN
writeln('Entrer une chaîne de caractères: ');
readln(ch);
aux:=ch[long(ch)];
ch[long(ch)]:=ch[1];
ch[1]:=x;
writeln('Voilà la chaîne cryptée: ', ch);
END.
\end{minted}
  \vspace{0.3cm}
  \hrule
  \vspace{0.3cm}

Définissez des constantes au début du programme c'est plus élégant.

\begin{tabular}{|c|}
	\hline
	Tableau de déclaration des nouveau types\\
	\hline\hline
	Eleves = tableau de 30 chaînes de caractères\\
	\hline
	Moyenne = tableau de 30 réels\\
	\hline
\end{tabular}


Le type scalaire par énumération défini un ensemble ordonné et fini de valeurs désignées par identificateur.
\end{document}
